\DeclareMathOperator{\logit}{logit}
\DeclareMathOperator{\Var}{Var}
\DeclareMathOperator{\SE}{SE}
\newcommand{\given}{\,|\,}
\newcommand{\indep}{\perp\!\!\!\!\perp}
\newcommand\diff{\mathop{}\!\mathrm{d}}
\newcommand\Diff[1]{\mathop{}\!\mathrm{d^#1}}
