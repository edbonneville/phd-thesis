%\usepackage[english]{babel}
%\usepackage[utf8]{inputenc}
%\usepackage{amsmath,amsfonts,amssymb}
%\usepackage{bm}
%\usepackage{amsmath}
%\usepackage[utf8]{inputenc}
%\usepackage{amsmath,amssymb,amsfonts,amsthm}

\KOMAoptions{%
  listof=totoc,
  listof=nochaptergap,
  listof=flat,
  index=totoc
}
%\usepackage{scrhack}

\DeclareMathOperator{\logit}{logit}
\DeclareMathOperator{\Var}{Var}
\DeclareMathOperator{\SE}{SE}
\newcommand{\given}{\,|\,}
\newcommand{\indep}{\perp\!\!\!\!\perp}
\newcommand\diff{\mathop{}\!\mathrm{d}}
\newcommand\Diff[1]{\mathop{}\!\mathrm{d^#1}}

% https://ctan.math.illinois.edu/macros/latex/contrib/defoldfonts/defoldfonts.pdf
% \usepackage[Sonny]{fncychap} % Only works with class: book
% https://github.com/komascript
% https://github.com/komascript/defoldfonts
%
%\usepackage[Sonny]{fncychap}
%\usepackage{scrlayer-fancyhdr}
\usepackage{scrlayer-scrpage}

% .. se defoldfonts documentation..
% .. is this an issue with the bibliography??
%\DeclareOldFontCommand{\rm}{\normalfont\rmfamily}{\mathrm}
%\DeclareOldFontCommand{\sf}{\normalfont\sffamily}{\mathsf}
%\DeclareOldFontCommand{\tt}{\normalfont\ttfamily}{\mathtt}
%\DeclareOldFontCommand{\bf}{\normalfont\bfseries}{\mathbf}
%\DeclareOldFontCommand{\it}{\normalfont\itshape}{\mathit}
%\DeclareOldFontCommand{\sl}{\normalfont\slshape}{\@nomath\sl}
%\DeclareOldFontCommand{\sc}{\normalfont\scshape}{\@nomath\sc}
%\DeclareOldFontCommand{\sfb}{\normalfont\sffamily\bfseries}{\@nomath\sfb}

%\usepackage[Sonny]{fncychap}
%\usepackage[Lenny]{fncychap}
%\usepackage[Bjornstrup]{fncychap}
%\KOMAoption{listof}{nochaptergap}
%\KOMAoptions{draft=true}

% --------------------------------------------------------------------------
% --------------------------------------------------------------------------
% Originalcode von: <http://www.komascript.de/fncychap-Lenny>
% Copyright (c) Markus Kohm
% Version: 2018-01-22
% Changes:
% - 2016-09-03 erste Version
% - 2017-05-24 Anpassung von beforeskip an aktuelles KOMA-Script
% - 2018-01-22 Korrektur der Liniendicke (Dank Elke)
% Weitergabe und Verwendung gestattet, solange dieser Hinsweiskommentar
% einschließlich Link und Copyrightinformation erhalten bleibt.

% 1. Emulation von fncychap mit KOMA-Script-Mitteln:
\newlength{\ChapterRuleWidth}
\newcommand*{\ChRuleWidth}[1]{\setlength{\ChapterRuleWidth}{\dimexpr #1}}%
\newcommand*{\ChNameVar}{\setkomafont{chapterprefix}}%
\newcommand*{\ChTitleVar}{\setkomafont{chapter}}%
\newcommand*{\ChNumVar}{\setkomafont{chapternumber}}%
\newcommand*{\ChapterNameCase}[1]{#1}
\newcommand*{\ChNameUpperCase}{\let\ChapterNameCase\MakeUppercase}
\newcommand*{\ChNameIs}{\renewcommand*\ChapterNameCase[1]{##1}}
\newcommand*{\ChNameLowerCase}{\let\ChapterNameCase\MakeLowercase}
\newcommand*{\ChapterTitleCase}[1]{#1}
\newcommand*{\ChTitleUpperCase}{\let\ChapterTitleCase\MakeUppercase}
\newcommand*{\ChTitleIs}{\renewcommand*\ChapterTitleCase[1]{##1}}
\newcommand*{\ChTitleLowerCase}{\let\ChapterTitleCase\MakeLowercase}

% 2. Einstellungen für den Stil Sonny:
\ChRuleWidth{1pt}
\KOMAoptions{chapterprefix}% Es ist ein Präfix-Stil
\newkomafont{chapternumber}{\fontsize{60}{62}\usefont{OT1}{ptm}{m}{n}\selectfont}
\RedeclareSectionCommand[%
  beforeskip=-61pt,% Abstand über der Präfixzeile bzw. der Linie
  %innerskip=15pt,% Abstand zwischen Präfixzeile und Text
  innerskip=20pt,% Abstand zwischen Präfixzeile und Text
  afterskip=40pt,% Abstand unter dem Text
  font=\normalfont\rmfamily\Huge,% Schrift des Namens
  prefixfont=\fontsize{14}{16}\usefont{OT1}{phv}{m}{n}\selectfont,% Schrift der Präfixzeile
]{chapter}
\usepackage{picture}
\usepackage{xcolor}
\renewcommand*{\chapterformat}{%
  \mbox{%
    \setlength{\fboxsep}{0pt}\colorbox{white}{%
      \strut\ChapterNameCase{\chapappifchapterprefix{\enskip}}}%
    {\usekomafont{chapternumber}{%
        \colorbox{white}{\strut\thechapter\IfUsePrefixLine{}{\enskip}}}}%
  }%
}
\renewcommand*{\chapterlineswithprefixformat}[3]{% Ebene, Nummer, Text
  \IfArgIsEmpty{#2}{}{%
    % Die Prefix-Zeile aus Argument 2 wird nur gesetzt, wenn sie vorhanden
    % ist.
    \begin{picture}(0,0)
      \setlength{\linethickness}{\ChapterRuleWidth}%
      \usekomafont{chapternumber}{%
      \put(.5\ChapterRuleWidth,0){%
        \framebox(\dimexpr\linewidth-\ChapterRuleWidth,.9\ht\strutbox){}}}%
    \end{picture}%
    #2%
  }%
  \ChapterTitleCase{#3}%
}
% --------------------------------------------------------------------------



%https://ctan.org/pkg/uni-titlepage?lang=en

% need thumb indices!!

% https://github.com/novoid/LaTeX-KOMA-template
% https://cameronpatrick.com/post/2023/07/quarto-thesis-formatting/#basic-principles
% fncychap koma https://tex.stackexchange.com/questions/195757/how-to-use-fncychaps-style-for-section
