% Code for chapter titles, with small adjustments/commments added

% Other notes:
% - The "Needed as minimum" tags were added when trying to make a minimal version
% of the below, but never continued in the end

% --------------------------------------------------------------------------
% --------------------------------------------------------------------------
% Originalcode von: <http://www.komascript.de/fncychap-Lenny>
% Copyright (c) Markus Kohm
% Version: 2018-01-22
% Changes:
% - 2016-09-03 erste Version
% - 2017-05-24 Anpassung von beforeskip an aktuelles KOMA-Script
% - 2018-01-22 Korrektur der Liniendicke (Dank Elke)
% Weitergabe und Verwendung gestattet, solange dieser Hinsweiskommentar
% einschließlich Link und Copyrightinformation erhalten bleibt.

% 1. Emulation von fncychap mit KOMA-Script-Mitteln:
\newlength{\ChapterRuleWidth}% -> Needed as minimum
\newcommand*{\ChRuleWidth}[1]{\setlength{\ChapterRuleWidth}{\dimexpr #1}}% -> Needed as minimum
\newcommand*{\ChNameVar}{\setkomafont{chapterprefix}}%
\newcommand*{\ChTitleVar}{\setkomafont{chapter}}%
\newcommand*{\ChNumVar}{\setkomafont{chapternumber}}%
\newcommand*{\ChapterNameCase}[1]{#1}% -> Needed as minimum
\newcommand*{\ChNameUpperCase}{\let\ChapterNameCase\MakeUppercase}
\newcommand*{\ChNameIs}{\renewcommand*\ChapterNameCase[1]{##1}}
\newcommand*{\ChNameLowerCase}{\let\ChapterNameCase\MakeLowercase}
\newcommand*{\ChapterTitleCase}[1]{#1}% -> Needed as minimum
\newcommand*{\ChTitleUpperCase}{\let\ChapterTitleCase\MakeUppercase}
\newcommand*{\ChTitleIs}{\renewcommand*\ChapterTitleCase[1]{##1}}
\newcommand*{\ChTitleLowerCase}{\let\ChapterTitleCase\MakeLowercase}

% 2. Einstellungen für den Stil Sonny:
\ChRuleWidth{1pt}
\KOMAoptions{chapterprefix}% Es ist ein Präfix-Stil
\newkomafont{chapternumber}{\fontsize{60}{62}\usefont{OT1}{ptm}{m}{n}\selectfont}
\RedeclareSectionCommand[%
  beforeskip=-61pt,% Abstand über der Präfixzeile bzw. der Linie
  innerskip=15pt,% Abstand zwischen Präfixzeile und Text
  afterskip=40pt,% Abstand unter dem Text
  % -> Below 2 lines commented to inherit the font of the document:
  %font=\normalfont\rmfamily\Huge % Font for Chapter text (not the number)
  %prefixfont=\fontsize{14}{16}\usefont{OT1}{phv}{m}{n}\selectfont,% Schrift der Präfixzeile
]{chapter}
\usepackage{picture}
\usepackage{xcolor}
\renewcommand*{\chapterformat}{%
  \mbox{%
    \setlength{\fboxsep}{0pt}\colorbox{white}{%
      \strut\ChapterNameCase{\chapappifchapterprefix{\enskip}}}%
    {\usekomafont{chapternumber}{%
        \colorbox{white}{\strut\thechapter\IfUsePrefixLine{}{\enskip}}}}%
  }%
}
\renewcommand*{\chapterlineswithprefixformat}[3]{% Ebene, Nummer, Text
  \IfArgIsEmpty{#2}{}{%
    % Die Prefix-Zeile aus Argument 2 wird nur gesetzt, wenn sie vorhanden
    % ist.
    \begin{picture}(0,0)
      \setlength{\linethickness}{\ChapterRuleWidth}%
      \usekomafont{chapternumber}{%
      \put(1.25\ChapterRuleWidth,0){% -> go from .5 -> 1.25 to remove annoying vertical thing on left of text
        \framebox(\dimexpr\linewidth-\ChapterRuleWidth,.9\ht\strutbox){}}}%
    \end{picture}%
    #2%
  }%
  \ChapterTitleCase{#3}%
}
% --------------------------------------------------------------------------
